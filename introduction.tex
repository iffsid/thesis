\chapter{Introduction}

\section{The Sentence Tracker}
We introduce a framework that integrates natural-language descriptions of
activities with a detection-based tracking mechanism for activity recognition
in a provably optimal fashion.
%
We intend for the sentences, describing interaction between participant
objects, to provide high-level top-down guidance to the inherent tracking
process in the activity recognition framework.
%
Given a video and a sentence as its input, it produces as output a score and a
set of tracks that best represent the provided sentential description.
%
By leveraging this framework in different ways, we are able to demonstrate its
efficacy and elegance in performing three disparate tasks: (a)
sentential-description based focus of attention, (b) sentential-description
generation for video, and (c) sentential-query based video retrieval.

\section{Video Retrieval}

We modify and extend the Sentence Tracker’s ability to perform query-based
video retrieval on a very large corpus of videos using machine-learning
methods in order to learns models for the query lexicon.
%
We demonstrate how the roles that objects/participants play in the described
activities affects the retrieval mechanism, and how the lack of such a
mechanism is detrimental to the retrieval task.
%
Also, our tests are carried out with the help of off-the-shelf object
detectors.
%
Moreover, we do so restricting our method to only use off-the-shelf pre-trained
object detectors from the current state-of-the-art.
%
We do so to draw focus to the benefits of our underlying framework, avoiding
the confounds of detectors trained in-house.
%
The scalability and efficacy of our framework is demonstrated by deploying its
capabilities across a corpus of ten full-length Hollywood videos, amounting to
approximately 25 hours of video, and comparing against existing
state-of-the-art methods to do video retrieval.
%
The incorporation of query semantics provides us with a distinct advantage in
such tasks.

\section{The \LincolnLogs\ Anthology}
This work involves reasoning about the physical structure of composable
entities, involving integration within vision, and across vision and language
through the medium of robotics.
%
The work is instantiated in the domain of \LincolnLogs, which allows for a
combinatorially large number of assemblies from a relatively tiny inventory of
1-, 2-, and 3-notch logs.
%
We estimate the composition of a given assembly of \LincolnLogs\ by reasoning
about the unreliable low-level visual features in the context of high-level
physical constraints of assembly (E.g., logs combine orthogonally and only at
notches).
%
The framework allows for reasoning about occlusion, reasoning about an assembly
from multiple views and partial disassembly, and reasoning about an assembly in
the context of natural-language descriptions of the structure.
%
The linguistic framework derives the lexicon to describe a \LincolnLog\
assembly from topology, Such integration in our framework is made easy by the
fact that, in practice, they only involve expanding and contracting the overall
constraint-satisfaction problem.

\section{Game-Learning and Language}
Inspired by how children appear to learn rules about the world from
demonstration and interaction, we model, in the domain of board games, an
analogous system that learns the rules of board games from visual
observation.
%
Here, two robotic agents, the protagonist and the antagonist play a physically
instantiated board game.
%
A third agent, the wannabe, watches the gameplay and attempts to infer the
rules of the game, given some minor background knowledge (what directions mean,
etc.) about the world.
%
The physical instantiation forces the inference process, driven by Inductive
Logic Programming (ILP), to happen from real-world input.
%
Using this frame- work, we learn the complete rules (initial board, legal move
generator, and outcome predicate) of six games -- Tic-Tac-Toe, Hexapawn and 4
variants thereof.
%
Furthermore, We introduce a natural-language component into the system, by
giving the protagonist and the antagonist the rules of the game being played in
English, and enabling translation from the internal representations to natural
language.

\section{Compositionality in the Brain}

We investigate the compositionality of argument structure, i.e., how
participants fill roles in events, of sentences in the human brain, in a
first-of-its-kind study where people in an fMRI machine are shown videos
depicting activities that correspond to unique sentential descriptions.
%
We recover the compositional-semantic components independently from
counterbalanced experiments on stimuli that vary multiple variables by
collapsing along different variables, showing that brain-activity patterns
reflect compositionality in sentence structure as the composition of
independent experiments matches those obtained jointly.
%
Additionally, we also produce the first result on the classification of verbs
in the human brain, performing with 80\% accuracy on a 1-out-of-6
classification task involving the picked up, put down, carry, hold, walk, and
dig events.
%
Furthermore, the robustness of these results were tested and verified by both
testing across subjects and across sites (to account for variance in
equipment).

\section{Real-time Action Recognition}
The central contribution of this body of work is the implementation and
deployment of a real-time multi-class activity and description system.
%
We make use of our viterbi-tracker framework for activity recognition, running
a multitude of such trackers in parallel on objects, using both our own
in-house object detector and the Felzenszwalb object detector optimized for the
GPU.
%
The in-house object detector is particularly effective in identifying rigid
objects under arbitrary spatial transformations, with only a handful of
training examples; a shortfall of existing state-of-the-art object detectors.
%
The system is capable of identifying actions such as picked up, put down,
raised, lowered, give, carry, walked, hold, replaced, and exchanged.
%
On identifying the action, the system produces natural-language output, both
orthographic and aural, describing any subset of the action, its constituents,
the roles played by such, spatial relations, and the temporal profile of the
action in sentential form.
