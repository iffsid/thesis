\begin{dedication}
  This is all a bit pointless.
\end{dedication}

\begin{acknowledgments}
  I'd like to thank my advisor J. M. Siskind, as well as my colleage and fellow
  graduate student Andrei Barbu, with whom I learnt a great deal and spent
  many a late night working on fun stuff.
  %
  I'd also like to thank my many great collaborators: D. Barrett, Z. Burchill,
  J. J. Corso, S. Dickinson, C. D. Fellbaum, S. H\'{e}lie, C. Hanson,
  S. J. Hanson, E. Malaia, B. A. Pearlmutter, T. M. Talavage, G. Tamer,
  J. Waggoner, S. Wang, R. B. Wilbur, C. Xiong, H. Yu, and others.  \vfill
  \begin{footnotesize}
    This work was supported, in part, by NSF grant CCF-0438806, by the Naval
    Research Laboratory under Contract Number N00173-10-1-G023, by the Army
    Research Laboratory accomplished under Cooperative Agreement Number
    W911NF-10-2-0060, and by computational resources provided by Information
    Technology at Purdue through its Rosen Center for Advanced Computing.
    %
    Any views, opinions, findings, conclusions, or recommendations contained or
    expressed in this document or material are those of the author(s) and do not
    necessarily reflect or represent the views or official policies, either
    expressed or implied, of NSF, the Naval Research Laboratory, the Office of
    Naval Research, the Army Research Laboratory, or the U.S. Government.
    %
    The U.S. Government is authorized to reproduce and distribute reprints for
    Government purposes, notwithstanding any copyright notation herein.
  \end{footnotesize}
\end{acknowledgments}

\tableofcontents

\listoftables

\listoffigures

\begin{abstract}
  Compositionality can be found almost everywhere one looks. It is manifest in as
  diverse a range of entities as objects around us, the languages we use, and
  even our actions and interactions with the world.
  %
  My work involves exploring and exploiting the general nature of such
  compositionality, often across multiple modalities such as vision and
  language, to solve deep and complex problems in perception.
  %
  I demonstrate such ability in a variety of domains including part-based
  structures, board games, and activity recognition.
  %
  I also show evidence for a particular kind of compositionality in how the
  brain perceives the world, lending further credence to the ubiquity and
  utility of compositionality.
\end{abstract}
